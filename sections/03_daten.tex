% !TEX root = ../main.tex
% Daten

\todo{Martin fragen nach Namen f"ur die "`Matrix"'}

Dieser Teil der Arbeit besch"aftigt sich mit der Struktur der genutzten Datens"atze.
Es wird erl"autert welche Schritte ben"otigt werden um die Datens"atze zu erstellen und nach welchen Regel sie formatiert sind.
Au"serdem wir beschrieben, welche Schritte n"otig sind um die n"otigen Informationen f"ur die Visualisierung zu extrahieren.
Bei den genutzte Datens"atzen handelt es sich um einen Wikipedidump im XML Format und um eine sp"arlich besetzte Adjazenzmatrix.
Beide Datens"atze sind sehr unterschiedlich in ihrer Struktur, als auch ihrem Inhalt und Umfang.
In den letzten beiden Teilen dieses Kapitels wird genauer auf den Aufbau der Datenbank eingegangen und wie die Daten w"ahren der Laufzeit des Programms bearbeitet werden.

%NOTE: vllt groesse martin erste simMatrix
\begin{table}
\centering
% \begin{center}
\begin{tabular}{l r}
  \hline
  Datensatz & Gr"o"se \\
  \hline
  Wikipediadump & 54 GB \\
  "Ahnlichkeitsmatrix & 642 GB \\
  \hline
\end{tabular}
% \end{center}
\caption{Gr"o"se der Datens"atze in GB}
\label{dataset-size}
\end{table}


\section{"Ahnlichkeitsmatrix}
Der Gr"o"sere von beiden Datens"atzen ist die d"unn besetzte Matrix.
In der Matrix werden die berechneten "Ahnlichkeitswerte zwischen zwei Artikeln gespeichert.

In der "Ahnlichkeitsmatrix nach \cite{riehmann2016visualizing} werden alle Artikel der Wikipedia mit 
In zwei vorangegangen Projekten wurde



\begin{itemize}
    \item Bezugname Arbeit von Potthast
    \item Bezugnahme zur Thesis von Tristan Licht
    \item Vergleich der zwei Datens"atze

    \begin{itemize}
      \item Verwendetes "Ahnlichkeitesma"s
      \item "Ahnlichkeit von Paragraphen
      \item Daten "uber die "Ahnlichkeiten (Anzahl)
    \end{itemize}
    \item Inhalt der Matrix
    \begin{itemize}
      \item
    \end{itemize}
\end{itemize}
 % =================================================



\section{Wikipedia XML Dump}
\begin{itemize}
    \item Datenstruktur des Wikipediakorpus
    \item Notwendige Metainformationen aus dem Korpus
    \item Struktur Kategoriebaum
\end{itemize}
 % =================================================



\section{Datenbankentwurf}
\begin{itemize}
    \item Erl"auterung der Datenbankstruktur
\end{itemize}
 % =================================================


\section{Ablauf der Datenverabeitung}
\begin{itemize}
    \item Aufbereitung der Rohdaten: S"auberung, Formatierung
    \item Erstellung der Datenbank
\end{itemize}
 % =================================================
