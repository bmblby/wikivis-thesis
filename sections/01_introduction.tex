% !TEX root = ../main.tex
% Einleitung


% einf"uhrung allgemein B"aume und Enzyklop"adien
Das Bed"urfniss der Veranschaulichung von Informationen reicht weit zur"ck in die Geschichte des Menschen.
Eine der "Altesten visuellen Methaphern ist dabei der Baum.
Darstellungen von B"aumen k"onnten unterschiedlicher nicht sein, von realistichen Abbildungen bis zu der abstrakten Form von simplen Knoten und Kanten.
Als visuelle Methaper um Informationen darzustellen, reichen die Bereiche, in welchen der Baum genutzt wird, von Erbreinfolgen oder der Hierarchie von Tieren.
Mit Hilfe des Baumes als visuelle Representation, lassen sich Komplexe Beziehungen zwischen einzelnen Entit"aten darstellen.\cite{lima2014book}
Viele Darstellungen von B"aumen machen sich vor allem ihre hierarhiche und geordnete Struktur zu nutzten um Wissen zu veranschaulichen.

% Genutzt f"ur viele verschiedene Darstellungen von Beziehungen, wie Erbreinfolge, Darstellung von Enzyklop"adien.
% Beispiel (Lima)

%NOTE
% Relevanz von Informationsvisualisierung\\
% Bezug auf das vorangegangene Projekt Visual Text Analytics\\
% einmal eine Abtrakte Motivation -> Darstellung Kategorienbaum + "Ahnlichkeiten
%
% Motivation aus dem vorg"anger Projekt -> konkreter Bezug auf fehlende eigenschaften und Erkenntnisse
% defacto die einzige Visualisierung f"ur den Kategoriebaum der Wikipedia
% =========================================================


\section{Motivation}
\begin{itemize}
  \item Bedeutung des Wikipediadatensatzes
  \item Kritikpunkte im Projekt Visual Text Analytics
  \begin{itemize}
    \item statische Layout
    \item fehlende M"oglichkeit zur Exploration der Daten
    \item "Ubersicht der Visualisierung ist sehr undurchsichtig
    \item Schwierigkeiten beim nachvollziehen der Verbindungen der Artikel untereinander
    \item "Anderung des Schwellwertes f"ur "Ahnlichkeiten -> zu viele Kanten
    \item fehlende M"oglichkeit die Auswahl an Artikel einzuschr"anken
    \item Unterst"utzende Funktion zum Zusammenfassen von mehreren Artikel -> Kategorien, Cluster
    \item Artikelcluster mit viele "Ahnlichkeiten zueinander sind schwer zu lesen
    \item Kategorienbaum nur unzureichend als Unterzt"utzung neben den Artikeln
  \end{itemize}
\end{itemize}

Der Fokus der Arbeit liegt darin, eine geeignete Darstellung der Vebindungen zwischen den Kategorien der Wikipedia zufinden.
Desweiteren setzt sich die Arbeit damit auseinander eine Strategie zu entwickeln, um eine d"unn besetzt Matrix in der Gr"o"se von 800GB zu veranschaulichen und zug"anglich zu machen.
Beide Teile der Arbeit haben als gemeinsames Ziel das Verst"ndnis der Wikipedia zu verbessern.


Ziel der Arbeit ist es, die hierarchiche Struktur der Kategorien aus der Wikipedia darzustellen.
Dabei soll es m"oglich sein jede Kategorie zu suchen und sich ihre Unterkategorien beliebig tief anzuschauen.
Wichtig ist vor allem, das beim Explorieren der Unterkategorien keine  Knoten oder Kanten "ubereinander gezeichnet werden.

Im  Projekt Visual Text Analysis


\section{Ziel}
\begin{itemize}
    \item Verbesserung des Verst"andnisses des "Ahnlichkeitsma"ses
    \item Darstellung der Hierarchie von Kategorien
    \begin{itemize}
      \item Konstruktion eines Kategorienbaum mit der M"oglichkeit der Exploration
    \end{itemize}
    \item Interaktive Visualisierung
    \begin{itemize}
      \item "Anderung des Schwellwertes zur Laufzeit
    \end{itemize}
    \item Exploration der Daten
    \item Bezug zu anderen Datens"atzen: Authoren, Editierungen, etc.
\end{itemize}














% % ////////////////////////////////////
% \begin{figure}[H]
%     \centering
%     \begin{subfigure}[b]{0.96\textwidth}
%         \includegraphics[width=\textwidth]{images/01_introduction/motivation_temporal_superresolution_one.pdf}
%         \caption{3D video playback rate matching the low recording rate}
%         % \label{subfig:playback_rate_matching}
%     \end{subfigure}

% 	\vspace{1.0cm}

%     \begin{subfigure}[b]{0.96\textwidth}
%         \includegraphics[width=\textwidth]{images/01_introduction/motivation_temporal_superresolution_two.pdf}
%         \caption{3D video playback matching the video encoding standards to ensure persistence of fluid motion}
%         % \label{subfig:playback_rate_higher}
%     \end{subfigure}
%     \caption[Concept of Temporal Super Resolution in a 3D Video Player]{In this Figure, two different playback rates for a low frame rate recording of a 3D volume sequence is presented. The tick marks on the black arrows indicate the position of recorded frames. The blue arrows indicate the playback rate that matches the recording \subref{subfig:playback_rate_matching} and an increased playback rate \subref{subfig:playback_rate_higher} using temporal interpolation to provide the viewer with a impression of fluid motion. }
%     % \label{fig:temporal_superresolution}
% \end{figure}

%  \cite{VillanuevaEtAl:2016}