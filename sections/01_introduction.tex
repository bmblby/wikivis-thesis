\section{Motivation}



































% ////////////////////////////////////
\begin{figure}[H]
    \centering
    \begin{subfigure}[b]{0.96\textwidth}
        \includegraphics[width=\textwidth]{images/01_introduction/motivation_temporal_superresolution_one.pdf}
        \caption{3D video playback rate matching the low recording rate}
        % \label{subfig:playback_rate_matching}
    \end{subfigure}
    
	\vspace{1.0cm}
    
    \begin{subfigure}[b]{0.96\textwidth}
        \includegraphics[width=\textwidth]{images/01_introduction/motivation_temporal_superresolution_two.pdf}
        \caption{3D video playback matching the video encoding standards to ensure persistence of fluid motion}
        % \label{subfig:playback_rate_higher}
    \end{subfigure}
    \caption[Concept of Temporal Super Resolution in a 3D Video Player]{In this Figure, two different playback rates for a low frame rate recording of a 3D volume sequence is presented. The tick marks on the black arrows indicate the position of recorded frames. The blue arrows indicate the playback rate that matches the recording \subref{subfig:playback_rate_matching} and an increased playback rate \subref{subfig:playback_rate_higher} using temporal interpolation to provide the viewer with a impression of fluid motion. }
    % \label{fig:temporal_superresolution}
\end{figure}

 \cite{VillanuevaEtAl:2016} 