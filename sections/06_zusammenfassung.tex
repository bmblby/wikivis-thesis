% Zusammenfassung

Diese Arbeit befasst sich mit der Darstellung der Kategorienhierarchie der Wikipedia und erarbeitet in diesem Zusammenhang eine Methode, die auf der Grundlage der Zugehörigkeit eines Artikels zu einer Wikipedia-Kategorie den Datensatz mit Ähnlichkeiten zwischen den Wikipedia-Artikeln filtert.

Die Analyse der Kategorienstruktur der Wikipedia ergibt, dass die Kategorien keiner strikten Hierarchie folgen.
Stattdessen stellt sich heraus, dass die Kategorienstruktur einem vernetzten, gerichteten Graphen von Themen gleicht und folglich die Themengebiete ebenfalls miteinander vernetzt sind.
Die Analyse zeigt darüber hinaus, dass eine inhaltliche, thematische Ordnung der Kategorien, welche die Grundlage für die Visualisierung der Kategorien in dieser Arbeit bildet, durchaus vorliegt.

Um diese Ordnung grafisch zu verdeutlichen, werden die Hierarchien in Form eines Baumes dargestellt.
Dafür wurde eine Methode entwickelt, bei der jede Kategorie als Ausgangspunkt für die Konstruktion eines Kategorienbaumes bestimmt werden kann.
Der Baum ist die visuelle Übersetzung einer vereinfachten Kategorienstruktur, welche wiederum eine Interpretation des Kategoriengraphen, abhängig von einem festgelegten Ausgangspunkt, ist.
Der gewählte Algorithmus garantiert, dass es innerhalb der Visualisierung zu keinen Überzeichnungen kommen kann.

Dem Nutzer werden für die Exploration des Kategorienbaums verschiedene interaktive Werkzeuge, wie die Erweiterung oder Suche, zur Verfügung gestellt. 
Während die Arbeit einerseits dem Nutzer eine übersichtliche Visualisierung liefert, beschäftigt sie sich andererseits mit der Handhabung der Daten, einem Bereich, der dem Nutzer nicht zugänglich ist.
Für das dargestellte Themengebiet wird eine Datenstruktur erschaffen, durch die eine erneute Traversierung bei zukünftigen Visualisierungen schneller umgesetzt werden kann.

Die Ähnlichkeitsmatrix aus der Arbeit von Licht~\cite{licht:2017} wird durch die \emph{kategorienbasierte} Filterung erheblich verkleinert und während der Laufzeit der Anwendung verfügbar gemacht.
Die Ähnlichkeitsmatrix lässt sich während der Exploration des Kategorienbaums über den Schwellwert für Ähnlichkeiten flexibel verkleinern.
Durch die zwei beschriebenen Methoden der Filterung wird die Anzahl der Ähnlichkeiten verringert und die Darstellung dieser vereinfacht.

% Während die inhaltliche Filterung, die Ähnlichkeiten innerhalb und zwischen Themengebieten aufzeigt, dem Nutzer zu einem höheren Grad an Verständnis über die zugrundeliegenden Artikelüberschneidungen in der Wikipedia verhilft, schafft die für diesen Kontext innovative Darstellung in Form eines Baumdiagramms ein neues Verständnis über die Kategorienstruktur auf visueller Ebene.

Die Zielsetzungen die aus dem Projekt~\emph{Visual Text Analytics} resultierten, wie die Darstellung der Hierarchie von Kategorien und die Verkleinerung der Ähnlichkeitsmatrix, werden durch diese Arbeit erreicht.

% \paragraph{Ausblick} \label{subchap:future}
Da in der vorliegenden Arbeit der Fokus auf der Darstellung der Kategorien liegt, könnten sich weiterführende Arbeiten mit der Darstellung von Ähnlichkeiten zwischen den Artikeln auseinandersetzen. 
Die erzeugte Ähnlichkeitsmatrix bildet die Grundlage für zukünftige Visualisierungen und weitere Forschungsvorhaben, die sich den Erweiterungen der Anwendung widmen.

Zur Weiterentwicklung dieser Arbeit wäre ein möglicher Ansatzpunkt, in Bezug auf die Inklusion der Ähnlichkeiten zwischen den Artikeln, das Zeichnen von Kanten zwischen den Kategorien, welche ähnliche Artikel beinhalten. 
Der Farbwert der Kante könnte zum Indiz für die Anzahl der sich ähnelnden Artikelpaare innerhalb der Kategorien werden.
Um die massiven Überzeichnungen, die durch die Verbindungen ähnlicher Artikel entstehen, zu reduzieren, sollte das Verfahren \emph{Hierarchical Edge Bundling} nach \cite{holten2006hierarchical} verwendet werden.
Durch die \emph{Hierarchcal-Edge-Bundling}-Technik lassen sich Ähnlichkeiten zwischen Kategorien erkennen.
Eine solche Darstellung könnte als Funktion implementiert werden, die der Nutzer während der Exploration des Kategorienbaums bei Bedarf aktiviert.

Weiterhin könnte es von Interesse sein, zu untersuchen, welche Ähnlichkeiten zwischen Artikeln existieren, die in einer Kategorie außerhalb des Kategorienbaums liegen.
Die Kategorien der \emph{globalen} Vergleiche von den Artikeln, welche in dem Abschnitt~\ref{subchap: filter-vis} beschrieben werden, geben Hinweise auf Themengebiete mit weiteren möglichen Ähnlichkeiten zum bereits erkundeten Themengebiet.
Zur Darstellung dieses Sachverhaltes wäre es hilfreich, die Titel der von den \emph{globalen} Vergleichen betroffenen Kategorien um den Kategorienbaum herum anzuordnen.
Auf diese Art könnte dem Nutzer ein Vorschlag für die Erkundung eines neuen Themengebietes gegeben werden.

Entscheidet sich denn der Nutzer, ein neues Themengebiet zu erkunden, sollte die Möglichkeit bestehen, zwei Kategorienbäume in der Visualisierung darzustellen.
Dabei könnte es von Interesse sein, die Gemeinsamkeiten der beiden Kategorienbäume aufzuzeigen, wie etwa die entsprechenden Artikel oder Kategorien.
Weiterhin sollten auch Ähnlichkeiten zwischen den Artikeln der beiden Kategorienbäume sichtbar sein.
Mit diesen Informationen könnten neue Beziehungen zwischen den Themengebieten gefunden werden.

Neben den Weiterentwicklungen hinsichtlich der Darstellungen von Artikelähnlichkeiten wäre die Visualisierung ähnlicher Textpassagen zwischen den Artikeln eine vertiefende Erweiterung.
Dann würden die Fließtexte zweier Artikel gezeigt und die Paragraphen mit der höchsten Ähnlichkeit hervorgehoben.
Diese Darstellung würde es ermöglichen, weitere Verfahren zur Textwiederverwendung auf die Artikel anzuwenden.
Eines dieser Verfahren könnte ein \emph{Text-Alignment}-Verfahren sein, mit dem ähnliche Abschnitte zwischen den Artikeln gefunden werden.

Abschließend lässt sich festhalten, dass die in der vorliegenden Arbeit entwickelte Anwendung zum einen als eine Visualisierung des Kategorienbaums genutzt werden kann, genauer, um Unterkategorien eines Themengebietes zu erkunden.
Zum anderen lässt sich die Anwendung als ein Werkzeug zur Erstellung einer gefilterten Ähnlichkeitsmatrix einsetzen.
Die im Rahmen der Arbeit geleisteten Fortschritte, wie die erzeugte Ähnlichkeitsmatrix als Grundlage zukünftiger Arbeiten, stellen somit einen neuen Beitrag zum Forschungsbereich der Enzyklopädie Wikipedia dar.

% \todo{FINAL-SATZ-ABSCHLUSS}
% Unseres Wissens nach wurden die Verbindungen zwischen Kategorien der Wikipedia in Form eines Baumes noch nicht dargestellt, und stellt 
% somit einen Beitrag zum Forschungsbereich der Enzyklopädie Wikipedia dar.





% Die aus dem Projekt~\emph{Visual Text Analytics} heraus formulierte Zielsetzung, die Darstellung der Kategorien während der Laufzeit der Anwendung zu verändern, wird durch diese Arbeit erfüllt.


