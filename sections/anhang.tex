% !TEX root = ../main.tex

% XML
\lstset{style=codestyle, language=XML}
\begin{lstlisting}[caption= Auszug einer Artikelseite im XML Format. Rot markierte Bereiche sind durch das XML Format extrahierbar. Blau markierte Bereiche sind im Flie"stext einer Wikipediaseite und m"ussen mit regul"aren Ausdr"ucken extrahiert werden., label=xml-artikel]
  <page>
    <title>!\fcolorbox{red}{white}{Autism}!</title>
    <ns>!\fcolorbox{red}{white}{0}!</ns>
    <id>!\fcolorbox{red}{white}{25}!</id>
    <revision>
      <id>\fcolorbox{red}{white}{717042201}</id>
      <parentid>717001444</parentid>
      <timestamp>2016-04-25T11:26:03Z</timestamp>
      <contributor>
        <username>Doc James</username>
        <id>3810835</id>
      </contributor>
      <comment>Reverted [[WP:AGF|good faith]] edits by [[Special:Contributions/Official9090|Official9090]] ([[User talk:Official9090|talk]]): Ref please. ([[WP:TW|TW]])</comment>
      <model>wikitext</model>
      <format>text/x-wiki</format>
      <text xml:space="preserve">
          {{Hatnote|This article is about the classic autistic disorder; some writers use the word ''autism'' when referring to the range of disorders on the [[autism spectrum]] or to the various [[pervasive developmental disorder]]s.&lt;ref name=Caronna/&gt;}}
          
          [...]
          
          !\fcolorbox{blue}{white}{[[Category:Autism| ]]}!
          !\fcolorbox{blue}{white}{[[Category:Communication disorders]]}!
          !\fcolorbox{blue}{white}{[[Category:Mental and behavioural disorders]]}!
          !\fcolorbox{blue}{white}{[[Category:Neurological disorders]]}!
          !\fcolorbox{blue}{white}{[[Category:Neurological disorders in children]]}!
          !\fcolorbox{blue}{white}{[[Category:Pervasive developmental disorders]]}!
          !\fcolorbox{blue}{white}{[[Category:Psychiatric diagnosis]]}!
      </text>
      <sha1>s63eaoxbv6r44zxfvr4d0kvgjebiys7</sha1>
    </revision>
  </page>
\end{lstlisting}

\begin{lstlisting}[caption= Auszug einer Kategorieseite im XML Format. Rot markierte Bereiche sind durch das XML Format extrahierbar. Blau markierte Bereiche sind im Flie"stext einer Wikipediaseite und m"ussen mit regul"aren Ausdr"ucken extrahiert werden., label=xml-kategorie]
  <page>
    <title>!\fcolorbox{red}{white}{Category:World War II}!</title>
    <ns>!\fcolorbox{red}{white}{14}!</ns>
    <id>!\fcolorbox{red}{white}{690451}!</id>
    <revision>
      <id>!\fcolorbox{red}{white}{709735395}!</id>
      <parentid>701195680</parentid>
      <timestamp>2016-03-12T19:29:53Z</timestamp>
      <contributor>
        <username>Giso6150</username>
        <id>16578927</id>
      </contributor>
      <comment>- 72 categories using [[WP:HC|HotCat]] per [[WP:DUPCAT]] / Each of these countries has its own &quot;X in World War II&quot; category under &quot;WWII by country&quot;; those are nested under &quot;World War II by country&quot;</comment>
      <model>wikitext</model>
      <format>text/x-wiki</format>
      <text xml:space="preserve">
        {{Use dmy dates|date=August 2011}}
        {{Commons category|World War II}}
        {{Portal|World War II|War|History}}
        {{Category diffuse}}
        {{CatRel|World War I|Interwar period|Warfare post-1945}}
        
        This is a root category that contains sub-categories of subject areas related to '''[[World War II]]''', a global conflict between what were known as the [[Allies of World War II|Allies]] and the [[Axis powers]] between 1 September 1939 and 2 September 1945.
        
        !\fcolorbox{blue}{white}{[[Category:Wikipedia categories named after wars]]}!
        !\fcolorbox{blue}{white}{[[Category:20th-century conflicts]]}!
        !\fcolorbox{blue}{white}{[[Category:1930s conflicts]]}!
        !\fcolorbox{blue}{white}{[[Category:1940s conflicts]]}!
        !\fcolorbox{blue}{white}{[[Category:Conflicts in 1939]]}!
        !\fcolorbox{blue}{white}{[[Category:Conflicts in 1940]]}!
        !\fcolorbox{blue}{white}{[[Category:Conflicts in 1941]]}!
        !\fcolorbox{blue}{white}{[[Category:Conflicts in 1942]]}!
        !\fcolorbox{blue}{white}{[[Category:Conflicts in 1943]]}!
        !\fcolorbox{blue}{white}{[[Category:Conflicts in 1944]]}!
        !\fcolorbox{blue}{white}{[[Category:Conflicts in 1945]]}!
        !\fcolorbox{blue}{white}{[[Category:Global conflicts]]}!
        !\fcolorbox{blue}{white}{[[Category:Modern Europe]]}!
        !\fcolorbox{blue}{white}{[[Category:The World Wars| ]]}!
      </text>
      <sha1>9cy6kvimcfc807ghnxqbgxyo9zr271d</sha1>
    </revision>
  </page>
\end{lstlisting}











% \begin{algorithm}[H]
% \SetAlgoLined
%     \SetKwProg{Fn}{Function}{}{}
%     \SetKwInOut{Input}{input}
%     \SetKwInOut{Output}{output}

    
% \Fn{compute\_cw\_offset(c$_i$, dims$_{codeword}$, $dims_{codebook}$)}{

% \Input{an integer $c_i$ indexing a codeword, the 3D codeword and codebook texture dimensions $dims_{codeword}$ and $dims_{codebook}$}
% \Output{a 3D integer vector $offset_{codeword}$ indicating the codeword offset in the texture}

% \BlankLine

% \BlankLine

% $num\_codewords_{xy}$ $\gets$ $dims_{codebook_{xy} }$ $\mathbf{div}$ $dims_{codeword_{xy} }$   \Comment{component-wise division}

% $num\_codewords_{slice}$ $\gets$ $num\_codewords_x$ * $num\_codewords_y$ \Comment{number per texture slice}

% \BlankLine

% $offset_{codeword{_z}}$ $\gets$ $c_i$ $\mathbf{div}$  $num\_codewords_{slice}$ \Comment{compute z offset of codeword}
 
 
% $c_i$ $\gets$ $c_i$ - $num\_codewords_{slice}$ \Comment{reduce index by z offset}
  
% $offset_{codeword{_y}}$ $\gets$ $c_i$ $\mathbf{div}$  $num\_codewords_{x}$ \Comment{compute y offset of
%  codeword}
 
% $offset_{codeword{_x}}$ $\gets$ $c_i$ $\mathbf{mod}$ num\_codewords$_{x}$ \Comment{compute x offset of codeword}  

% \BlankLine

% \Return $offset_{codeword}$
 
% }
%  \caption{Compute Integer Codeword Offset}
%  \label{algo:codeword_offset}
% \end{algorithm}


% \newpage


% \begin{algorithm}[H]
% \SetAlgoLined
%     \SetKwProg{Fn}{Function}{}{}
%     \SetKwInOut{Input}{input}
%     \SetKwInOut{Output}{output}

    
%     %dims$_{idx}$, dims$_{codeword}$, $dims_{codebook}$)

% \Fn{transform\_to\_ncs($p$,  $dims_{idx\_vol}$)} {
% \Input{ a sampling position $p$ in normalized volume space, the index volume dimension
% s $dims_{idx\_vol}$}

% \Output{a sampling position relative to the dimensions of a codeword}

% \BlankLine

% \BlankLine

% $p_{abs_idx}$ $\gets$ $p$ * $dims_{idx\_vol}$ \Comment{express position absolute number of codewords covered}

% $p_{ncs}$ $\gets$ $get\_fractional(p_{abs_idx})$ \Comment retrieve fractional part

% \Return $p_{ncs}$

% }


% \Fn{sample\_vq\_element($p$, $\hat{V}_{idx}$, $\hat{V}_{cb}$, $dims_{codeword}$) }{

% \Input{ a sampling position $p$ in normalized volume space, the index volume $\hat{V}_{idx}$ and corresponding 3D codebook $\hat{V}_{cb}$, the dimensions of a 3D codeword $dims_{codeword}$ }

% \Output{a decompressed sample $\hat{V}(p)$ being the reconstruction of $V(p)$}

% \BlankLine

% \BlankLine

% $dims_{idx\_vol}$ $\gets$ $get\_dimensions(\hat{V}_{idx}$) \Comment{retrieve index volume dimensions}

% $dims_{codebook}$ $\gets$ $get\_dimensions(\hat{V}_{cb}$) \Comment{retrieve codebook dimensions}

% \BlankLine

% $c_i$ $\gets$ $sample\_texture\_normalized(\hat{V}_{idx}$, $p$) \Comment{sample codeword idx in $\hat{V}_{idx}$ at $p$}

% $offset_{codeword}$ $\gets$ $compute\_cw\_offset$($c_i$, $dims_{codeword}$, $dims_{codebook}$) \Comment{get int offset}

% $offset_{element_l}$ $\gets$ $transform\_to\_ncs$($p,  dims_{idx\_vol}$) \Comment{get local offset to cw element}

% $offset_{element_g}$ $\gets$ $offset_{codeword}$ + $offset_{element_l}$ \Comment{compute global offset}

% \BlankLine

% $vq\_decompressed\_sample$ $\gets$ $sample\_texture\_unnormalized$($\hat{V}_{cb}$, $offset_{element_g}$)

% \BlankLine

% \Return $vq\_decompressed\_sample$

 
% % \While{While condition}{
%  % instructions\;
%  % \eIf{condition}{
%   % instructions1\;
%   %instructions2\;
%   %}{
%   %instructions3\;
%   %}
%  %}
% }
%  \caption{Sample Vector Quantized Element}
%  \label{algo:vector_quantization_sampling}
% \end{algorithm}







% %  vec3 base_offset_plus_overflow = base_offset_fetch + overflow;



% %  if( base_offset_plus_overflow.x >= current_index_texture_size.x ) {
% %    overflow.x = 0;
% %    padded_remaining_pos.x = clamped_reset_pos;
% %  }

% %  if( base_offset_plus_overflow.y >= current_index_texture_size.y) {
% %    overflow.y = 0;
% %    padded_remaining_pos.y = clamped_reset_pos;
% %  }

% %  if( base_offset_plus_overflow.z >= current_index_texture_size.z) {
% %    overflow.z = 0;
% %    padded_remaining_pos.z = clamped_reset_pos;
% %  }


% %  vec3 final_index_pos = base_offset_fetch + overflow;
% %  final_index_pos = round(clamp(final_index_pos, vec3(0.0, 0.0, 0.0), vec3( current_index_texture_size ) - 1.0 ) )  ;
% %  ivec3 int_index_pos = ivec3(final_index_pos);





